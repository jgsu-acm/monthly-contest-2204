\documentclass{ctsol}

\title{ACM 算法与微应用实验室 2022 年 4 月月赛题解}
\date{2022 年 5 月 1 日}

\begin{document}

\maketitle
\addsolution{FMVP}{AgOH}{排序}
\addsolution{最强阵容}{AgOH}{分数规划}
\addsolution{八保舟}{AgOH}{莫队 + 值域线段树}
\addsolution{自古红蓝}{AgOH}{动态规划 + 矩阵加速}
\addsolution{5Yqg5a+G6YCa5L+h}{AgOH}{分层图最短路}
\addsolution{K-prime seive 2}{Tifa}{线性筛}

\section*{题目概览}

\solutiontab

\makesolution
\section*{做法}

签到题

\section*{标程}

\std{A}

\makesolution
\section*{做法}

本题这种类型的题目叫做分数规划,通解是二分答案

发现题目要求即:在所有 $(a_i,b_i)$ 中选出 $5$ 对:$(a_j,b_j)$,使得 $\cfrac{\sum_{i=j}^{5}a_j}{\sum_{i=j}^{5}b_j}$ 最大

我们发现最大值这个东西是有单调性的,显然可以二分答案,而二分答案的关键在于 \verb|check| 函数的实现,设我们要判断的二分值为 $m$,则有:

$$
    \begin{aligned}
                    & \cfrac{\sum_{i=j}^{5}a_j}{\sum_{i=j}^{5}b_j} > m \\
        \Rightarrow & \sum_{i=j}^{5}a_j - m\sum_{i=j}^{5}b_j > 0 \\
        \Rightarrow & \sum_{i=j}^{5}a_j-mb_j > 0
    \end{aligned}
$$

也就是说只要我们能在所有 $(a_i,b_i)$ 中选出 $5$ 个 $(a_j,b_j)$ 使得 $\sum_{i=j}^{5}a_j-mb_j > 0$,那么我们的二分值 $m$ 就是满足条件的

二分实数答案要注意精度问题(eps)

\section*{标程}

\std{B}

\makesolution
\section*{做法}

\section*{标程}

\std{C}

\makesolution
\section*{做法}

方案数,又不像是组合数学,考虑动态规划

若最后一个格子未涂色,则倒数第二个格子可涂可不涂;若最后一个格子涂色了,则倒数第二个格子一定不能涂色,于是:

\begin{itemize}
    \item 状态设计:$dp[i]$ 代表已考虑了前 $i$ 个格子的方案数
    \item 初始状态:$dp[1] = 2,~dp[2] = 3$
    \item 转移方程:$dp[i] = dp[i-1] + dp[i-2]$
    \item 所求结果:$dp[n]$
\end{itemize}

即斐波那契数列,注意到 $n \leq {10}^{18}$ 很大,需要使用矩阵加速

\section*{标程}

\std{D}

\makesolution
\section*{做法}

\section*{标程}

\std{E}

\makesolution
\section*{做法}

\section*{标程}

\std{F}

\end{document}
