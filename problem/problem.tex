\documentclass{ctpro}

\title{ACM 算法与微应用实验室 2022 年 4 月月赛题目}
\date{2022 年 4 月 30 日}

\begin{document}

\maketitle
\addproblem{FMVP}{1000}{256}{传统}{AgOH}
\addproblem{最强阵容}{1000}{256}{传统}{AgOH}
\addproblem{八保舟}{1000}{256}{传统}{AgOH}
\addproblem{自古红蓝}{1000}{256}{传统}{AgOH}
\addproblem{e}{1000}{256}{传统}{AgOH}
\addproblem{f}{1000}{256}{传统}{Tifa}

\section*{比赛信息}

\ctinfotab{ACM 个人赛}{C/C++,~Python,~Java}{3}

\section*{题目概况}

\problemtab

\section*{编译命令}

参见 OJ 帮助

\section*{注意事项}

\begin{itemize}
    \item C/C++ 中函数 \verb|main()| 的返回值类型必须是 \verb|int|,程序正常结束时的返回值必须是 $0$。
    \item C/C++ 代码必须完全符合 GNU C/C++ 标准,不能使用诸如绘图、Win32API、中断调用、硬件操作或与操作系统相关的API。
    \item C/C++ 代码中允许使用 STL 类库。
\end{itemize}

\paragraph*{} 祝大家取得好成绩!

\makeproblem
\section*{题目描述}

在 4 月 26 日刚刚结束的 CBA 总决赛中,辽宁本钢队以 4:0 的大比分击败了浙江广厦队,获得了队史第二座总冠军奖杯。

根据 CBA 联赛的规则,总决赛系列赛中场均 SPR 值最高的球员将会获得总决赛 MVP 的称号。现给出每名球员每场比赛的 SPR 值,请你帮 CBA 联盟评选出谁是总决赛 MVP。

\section*{输入格式}

第一行,一个整数 $n$,代表共有 $n$ 名球员有资格评选总决赛 MVP。

接下来 $n$ 行,每行一个字符串和四个实数,依次代表球员的名字和其总决赛每场比赛的 SPR 值。

数据保证:

\begin{itemize}
    \item 球员名字长度不超过 $20$,且内部不存在空格;
    \item SPR 值的绝对值不超过 $20$;
    \item 不会出现两个场均 SPR 值相同的球员。
\end{itemize}

\section*{输出格式}

一行,一个字符串,格式为:\texttt{x is the FMVP!},\texttt{x} 为总决赛 MVP 球员的名字。

\section*{输入输出样例}

\testcasetab
{
    10\par
    FuHao 10.5 5.8 -1.7 6.1\par
    ZhaoJiwei 4.8 7.4 5.4 18.9\par
    ZhouJuncheng 3.8 2.5 0 -2.3\par
    ZhangZhenlin 3.1 5.6 5.7 3.5\par
    CongMingchen 1.8 1.3 -2.4 0\par
    GuoAilun 6.9 -8.3 7.6 0.1\par
    KyleFogg 2.2 15.5 -0.1 3.8\par
    LiXiaoxu 2.4 -0.8 1.1 4.5\par
    EricMoreland 5.7 10.3 6.8 5.5\par
    HanDejun 0 3.3 6.4 4.4
}
{
    ZhaoJiwei is the FMVP!
}

\clearpage
\section*{说明/提示}

【样例解释】

球员 ZhaoJiwei 的场均 SPR 值为:$\cfrac{4.8+7.4+5.4+18.9}{4} = 9.125$,为场均 SPR 最高的球员,故球员 ZhaoJiwei 为总决赛 MVP。

\makeproblem
\section*{题目描述}

国际篮球联合会(FIBA)于 2020 年发布的规则的第 1.1 条部分内容如下:

\begin{quote}
    \textbf{1.1 Basketball game}

    Basketball is played by 2 teams of 5 players each.
\end{quote}

即篮球比赛是由两队各派出五人进行对垒。

每名球员有两个能力值:进攻效率和防守效率。一个五人阵容的实力为 $\cfrac{\text{五人进攻效率之和}}{\text{五人防守效率之和}}$。

给定辽宁本钢队所有球员的进攻效率和防守效率,现在需要在所有球员中选出五个人组成一个五人阵容,请问这个阵容的实力的最大值是多少?

\section*{输入格式}

第一行,一个整数 $n~(5 \leq n \leq 2 \times {10}^5)$,代表辽宁本钢队共有多少名球员。

接下来 $n$ 行,每行两个整数 $a_i,b_i~(1 \leq a_i,b_i \leq {10}^3)$,分别代表一名球员的进攻效率和防守效率。

\section*{输出格式}

一行,一个实数,代表能选出的实力最大的阵容的实力,保留 $6$ 位小数。

\section*{输入输出样例}

\testcasetab
{
    7\par
    2 3\par
    3 2\par
    1 8\par
    5 9\par
    7 3\par
    6 6\par
    3 1
}
{
    1.400000\par
}

\section*{说明/提示}

【样例解释】

最优选法为选择第 $1$、$2$、$5$、$6$、$7$ 名球员。

\makeproblem
\section*{题目描述}

史尔特尔非常喜欢吃冰淇淋,夏天到了,她缠着博士给她买了好多好多个冰淇淋。

每个冰淇淋都有一个快乐值 $h_i$,如果史尔特尔吃掉了这个冰淇淋,那么她自己的快乐值就会加上这个冰淇淋的快乐值。

史尔特尔只能一个一个地吃冰淇淋,但是夏天很热,冰淇淋会很快地融化,史尔特尔每吃掉一个冰淇淋,其他冰淇淋就会因为融化导致其快乐值 $-1$,每个冰淇淋的快乐值最小只会扣减到 $0$。

博士想让史尔特尔变得非常高兴,这样才能让她接受被医疗干员簇拥着永续黄昏的任务。史尔特尔初始的快乐值为 $42$,博士会给你 $q$ 次询问,每次询问一个区间 $[l,r]$,你需要帮博士算出只给史尔特尔吃第 $l,l+1,\cdots,r$ 个冰淇淋后史尔特尔能达到的最大快乐值是多少。

\section*{输入格式}

第一行,一个整数 $n,q~(1 \leq n,q \leq {10}^4)$,分别代表共有 $n$ 个冰淇淋以及博士的询问个数。

第二行,$n$ 个整数 $h_i~(1 \leq h_i \leq {10}^5)$,代表每个冰淇淋的快乐值。

接下来 $q$ 行,每行两个整数 $l,r~(1 \leq l \leq r \leq n)$,代表博士的一次询问。

\section*{输出格式}

共 $q$ 行,第 $i$ 行为博士的第 $i$ 次询问的答案。

\section*{输入输出样例}

\testcasetab
{
    5 5\par
    1 3 5 7 9\par
    1 3\par
    2 4\par
    3 5\par
    1 4\par
    2 5
}
{
    7\par
    12\par
    18\par
    12\par
    18
}

\makeproblem
\section*{题目描述}

给定一个 $1 \times n$ 的矩形格条,你需要对每个格子涂上红色或者蓝色。但是两个红色的格子不允许相邻,请问共有多少种涂色的方案?

因为答案可能很大,你只需要输出答案对 $998244353$ 取模的结果即可。

\section*{输入格式}

一行,一个整数 $n~(1 \leq n \leq {10}^{18})$。

\section*{输出格式}

一行,一个整数,代表答案。

\section*{输入输出样例}

\testcasetab
{
    3
}
{
    5
}

\testcasetab
{
    2333333
}
{
    394054407
}

\section*{说明/提示}

【样例解释 \#1】

$5$ 种合法的涂色方法分别为:红蓝红,红蓝蓝,蓝红蓝,蓝蓝红,蓝蓝蓝。

\makeproblem
\section*{题目描述}

\section*{输入格式}

\section*{输出格式}

\section*{输入输出样例}

\testcasetab
{

}
{

}

\makeproblem
\section*{题目描述}

\section*{输入格式}

\section*{输出格式}

\section*{输入输出样例}

\testcasetab
{

}
{

}

\end{document}
